% ---------------------------------------------------------------------
% EG author guidelines plus sample file for EG publication using LaTeX2e input
% D.Fellner, v1.17, Sep 23, 2010


\title[Peridynamics-Based Fracture Animation for Elastoplastic Solids]%
      {Peridynamics-Based Fracture Animation for Elastoplastic Solids}

% for anonymous conference submission please enter your SUBMISSION ID
% instead of the author's name (and leave the affiliation blank) !!
\author[W. Chen et al.]{W. Chen, F. Zhu, J. Zhao,
  S. Li, G. Wang}

%\author[D. Fellner \& S. Behnke]
%       {D.\,W. Fellner\thanks{Chairman Eurographics Publications Board}$^{1,2}$
%        and S. Behnke$^{2}$
%%        S. Spencer$^2$\thanks{Chairman Siggraph Publications Board}
%        \\
%% For Computer Graphics Forum: Please use the abbreviation of your first name.
%         $^1$TU Darmstadt \& Fraunhofer IGD, Germany\\
%         $^2$Institut f{\"u}r ComputerGraphik \& Wissensvisualisierung, TU Graz, Austria
%%        $^2$ Another Department to illustrate the use in papers from authors
%%             with different affiliations
%       }

% ------------------------------------------------------------------------

% if the Editors-in-Chief have given you the data, you may uncomment
% the following five lines and insert it here
%
% \volume{27}   % the volume in which the issue will be published;
% \issue{1}     % the issue number of the publication
% \pStartPage{1}      % set starting page

%-------------------------------------------------------------------------
\begin{document}

\maketitle

%-------------------------------------------------------------------------
\section{Revision Summary}

We have made some serious changes to our method and the manuscript according to the suggestions of the reviewers. The major changes include (but not limited to):
\begin{enumerate}
\item{blabla}
\item{bla}
\item{bla}
\item{bla}
\end{enumerate}

\section{Answers to Reviewers' Comments}

Here we respond to the comments made by each reviewer.

\noindent{}\textbf{Reviewer \#1:}

\noindent{Q: }\emph{Lack examples that show handling of crack tips.}

blbabla.

\noindent{Q: }\emph{The glass fracture example reveals fracture 'dust' at the particle level. Is it due to the ''numerical fracture``?}

blabla.

\noindent{Q: }\emph{Show compelling examples that demonstrate the benefits of the peridynamics approach.}

blabla.

\noindent{}\textbf{Reviewer \#2:}

\noindent{Q: }\emph{More explanation of Equation 1 is required.}

blabla.

\noindent{Q: }\emph{More motivation for why the different terms (equations 3-15) have the form that they do is required.}

blabla.

\noindent{Q: }\emph{More explanation of the parameter that controls the amount of maximum plasticity is required.}

blabla.

\noindent{Q: }\emph{Justification of Equation 18 is required.}

blabla.

\noindent{Q: }\emph{More comparisons against FEM are preferred.}

blabla.

\noindent{}\textbf{Reviewer \#3:}

\noindent{Q: }\emph{Use shorter notations for $A_{deviatoric}$ and $A_{dialatational}$.}

blabla.

\noindent{Q: }\emph{Provide a step overview of the computations performed in each time step.}

blabla.

\noindent{Q: }\emph{A supplementary document or appendix is required that shows one can derive the peridynamics constitutive model from a classic hyperelastic energy density.}

blabla

\noindent{Q: }\emph{Provide a discussion on the choice motivations for linear elastic model. Why not more general nonlinear models? Is there any technical difficulty?}

blabla

\noindent{Q: }\emph{In a lot of the elastic examples the motion is significantly damped. Why?}

blabla.

\noindent{Q: }\emph{How is the time integration of the governing equation discretized.}

blabla.

\noindent{Q: }\emph{A discussion on element inversion is preferred.}

blabla.

\noindent{}\textbf{Reviewer \#4:}

\noindent{Q: }\emph{The exposition can be improved by providing examples/comparisons that show the advantage of peridynamics over existing methods.}

blabla.

\noindent{Q: }\emph{Comparisons against FEM with more complex examples are required.}

blabla.

\noindent{Q: }\emph{Do the FEM simulations use the same material parameters as the peridynamics simulations for achieving the same effiect?}

blabla.

\noindent{Q: }\emph{Crack surfaces are jagged, why not smooth these surfaces or alternatively use a finer mesh?}

blabla.


%-------------------------------------------------------------------------
\bibliographystyle{eg-alpha}
%\bibliographystyle{eg-alpha-doi}

%\bibliography{paper-bib}

%-------------------------------------------------------------------------
%\newpage

\end{document}
