\documentclass[11pt,fullpage]{article}
\usepackage{color, soul}
\usepackage{multicol,amsmath,amssymb,algorithmic}
\usepackage{graphics,graphicx}
\usepackage[right = 2.5cm, left=2.5cm, top = 2.5cm, bottom =2.5cm]{geometry}
\usepackage{enumerate}
\pagestyle{plain}
\def\urlfont{\DeclareFontFamily{OT1}{cmtt}{\hyphenchar\font='057}
              \normalfont\ttfamily \hyphenpenalty=10000}

%\input macros

\newcommand{\subheading}[1]{\noindent \textbf{#1}}
\newcommand{\grad}{\nabla}
\newcommand{\jump}[1]{[#1]}
\newcommand{\limit}[2]{\lim_{#1 \rightarrow #2}}
\newcommand{\mb}[1]{\mathbf{#1}}
\newcommand{\reals}[1]{\mathbb{R}^{#1}}
\newcommand{\blue}[1]{{\color{blue}{{#1}}}}

\title{Peridynamics-Based Fracture Animation for Elastoplastic Solids:\\
      Supplementary Technical Document}

\begin{document}
\maketitle

\section{Introduction}

This document presents the derivation of the constitutive model proposed in the paper. Section \blue{\ref{section:2}} introduces some preliminaries for the derivation and explains why the classical continuum elasticity theory could be regarded as a special case of peridynamics. Section \blue{\ref{section:3}} lists the common procedures to derive the peridynamics formulation of a general hyperelastic constitutive model in continuum mechanics. Finally, section \blue{\ref{section:4}} presents the derivation of the model introduced in the paper, which is a \emph{nonlocal} extension of the linear elastic model in the \emph{local} theory.

\section{Preliminaries}\label{section:2}

\subsection{Peridynamics equations of motion and the discretization}

As is introduced in the paper, the peridynamics governing equation for any material point located at $\mb{x}$ is formulated as below:
\begin{equation}
\rho\ddot{\mb{u}}(\mb{x}) = \int_{H_\mb{x}}[\mb{T}<\mb{x}'-\mb{x}> - \mb{T}<\mb{x}-\mb{x}'>]dV_{\mb{x}}+\mb{b}(\mb{x}).
\label{eq:1}
\end{equation}
The state of material point at $\mb{x}$ is influenced by the possibly infinite number of material points $\mb{x}'$ that belong to its family $H_\mb{x}$. When the continuum is discretized into particles, the integral in Equation \blue{\ref{eq:1}} is replaced with the summations of particles within the horizon $\delta_\mb{x}$:
\begin{equation}
\rho\ddot{\mb{u}}(\mb{x}) = \sum_{\mb{x}',\mb{x}'\in H_\mb{x}}[\mb{T}<\mb{x}'-\mb{x}> - \mb{T}<\mb{x}-\mb{x}'>]V_{\mb{x}'}+\mb{b}(\mb{x}),
\label{eq:2}
\end{equation}
where $V_{\mb{x}'}$ is the volume of the discrete particle and its value depends on the distribution of particles.
\subsection{Peridynamics for local interactions}

In the limiting case where the horizon $\delta_\mb{x}$ approaches 0, the material point $\mb{x}$ interacts only with its immediate neighbors. See Figure \blue{\ref{fig:1}}, the material point with label $k$ interacts with the other six material points in the immediate vicinity denoted as $(k-l)$,$(k+l)$,$(k-m)$,$(k+m)$,$(k-n)$,and $(k+n)$. This conforms to the classical continuum mechanics, see the book \blue{\cite{bonet2008nonlinear}} for reference.
\begin{figure}[h]
  \centering
  \includegraphics[width=0.5\linewidth]{./fig1.png}
  \caption{\label{fig:1}
  Material point $k$ interacts with its immediate neighborhood. Image from \blue{\cite{madenci2014peridynamic}}.
}
\end{figure}

In the context of local interactions, Equation \blue{\ref{eq:2}} for material point $k$ is represented in the following form:
\begin{equation}
\rho_{(k)}\ddot{\mb{u}}_{(k)} = \sum_{j=k-l,k+l,k-m,k+m,k-n,k+n}(\mb{t}_{(k)(j)}-\mb{t}_{(j)(k)})V_{(j)} + \mb{b}_{(k)},
\label{eq:3}
\end{equation}
where $\mb{t}_{(k)(j)}$ denotes the internal force density that material point $j$ exerted on point $k$, and $\mb{t}_{(j)(k)}$ is the other way around.

Let's recall the form of governing equations in continuum mechanics:
\begin{equation}
\rho_{(k)}\ddot{\mb{u}}_{(k)} = \nabla\cdot\sigma_{(k)} + \mb{b}_{(k)},
\label{eq:4}
\end{equation}
where $\sigma_{(k)}$ is the cauchy stress at point $k$. We could further write it in a component-wise form and approximate the spatial derivatives with central difference. Take the $x$-component as an example:
\begin{equation}
\begin{aligned}
\rho_{(k)}\ddot{\mb{u}}_{x(k)} &= \quad \frac{1}{2}\frac{\sigma_{xx(k)} - \sigma_{xx(k-l)}}{\Delta x} + \frac{1}{2}\frac{\sigma_{xx(k+l)} - \sigma_{xx(k)}}{\Delta x}\\
                              &\quad+  \frac{1}{2}\frac{\sigma_{xy(k)} - \sigma_{xy(k-m)}}{\Delta y} + \frac{1}{2}\frac{\sigma_{xy(k+m)} - \sigma_{xy(k)}}{\Delta y}\\
                              &\quad+  \frac{1}{2}\frac{\sigma_{zx(k)} - \sigma_{zx(k-n)}}{\Delta z} + \frac{1}{2}\frac{\sigma_{xz(k+n)} - \sigma_{xz(k)}}{\Delta z}\\
                              &\quad + \mb{b}_{x(k)}.
\end{aligned}
\label{eq:5}
\end{equation}
Each term in the above equation involves only material point $k$ and one immediate neighbor. We could recast Equation \blue{\ref{eq:3}} in a similar form:
\begin{equation}
\begin{aligned}
\rho_{(k)}\ddot{\mb{u}}_{(k)} &= \quad (\mb{t}_{(k)(k-l)} - \mb{t}_{(k-l)(k)})V_{(k-l)} + (\mb{t}_{(k)(k+l)} - \mb{t}_{(k+l)(k)})V_{(k+l)}\\
                             &\quad + (\mb{t}_{(k)(k-m)} - \mb{t}_{(k-m)(k)})V_{(k-m)} + (\mb{t}_{(k)(k+m)} - \mb{t}_{(k+m)(k)})V_{(k+m)}\\
                             &\quad + (\mb{t}_{(k)(k-n)} - \mb{t}_{(k-n)(k)})V_{(k-n)} + (\mb{t}_{(k)(k+n)} - \mb{t}_{(k+n)(k)})V_{(k+n)}\\
                             &\quad  + \mb{b}_{(k)}.
\end{aligned}
\label{eq:6}
\end{equation}
By equating Equation \blue{\ref{eq:5}} (and its counterparts) with Equation \blue{\ref{eq:6}}, we get the relationships between the cauchy stress and the peridynamics internal force density:
\begin{equation}
\sigma_{\alpha\beta(k)} = 2t_{\beta(k)(q_\alpha)}\Delta\alpha V_{(q_\alpha)}\qquad \mathrm{with } \quad q_x=(k+l),q_y=(k+m),q_z=(k+n)
\label{eq:7}
\end{equation}
\begin{equation}
\sigma_{\alpha\beta(k)} = -2t_{\beta(k)(q_\alpha)}\Delta\alpha V_{(q_\alpha)}\qquad \mathrm{with } \quad q_x=(k-l),q_y=(k-m),q_z=(k-n),
\label{eq:8}
\end{equation}
where $\alpha,\beta=x,y,z$. Especially, for normal stress, we would have:
\begin{equation}
\sigma_{\alpha\alpha} = 2 \mb{t}_{(k)(q_\alpha)}\cdot(\mb{x}_{(q_\alpha)}-\mb{x}_{k})V_{(q_\alpha)}.
\label{eq:9}
\end{equation}
The subsequent equation would also be useful for the derivation in following sections:
\begin{equation}
\begin{aligned}
\sum_{\beta=x,y,z}\sigma_{\alpha\beta(k)}^2 &= \sum_{\beta=x,y,z}4t_{\beta(k)(q_\alpha)}^2(\Delta\alpha)^2 V_{(q_\alpha)}^2 \\
                                            &= 4(\mb{t}_{(k)(q_\alpha)}|\mb{x}_{(q_\alpha)}-\mb{x}_{k}|V_{(q_\alpha)})
                                                 \cdot
                                                (\mb{t}_{(k)(q_\alpha)}|\mb{x}_{(q_\alpha)}-\mb{x}_{k}|V_{(q_\alpha)}).
\end{aligned}
\label{eq:10}
\end{equation}

\subsection{Strain energy density and internal force density}

In continuum mechanics the hyperelastic constitutive models are generally represented as strain energy densities. Therefore it is necessary to define the relationship between strain energy density and peridynamics internal force density. According to the book by Madenci and Oterkus\blue{\cite{madenci2014peridynamic}}, in ordinary state-based peridynamics the relationship is defined as below:
\begin{equation}
\mb{t}_{(k)(j)}=\frac{1}{V_{(j)}}\frac{\partial W_{(k)}}{\partial (|\mb{y}_{(j)}-\mb{y}_{(k)}|)}\frac{\mb{y}_{(j)}-\mb{y}_{(k)}}{|\mb{y}_{(j)}-\mb{y}_{(k)}|},
\label{eq:11}
\end{equation}
in which $W_{(k)}$ is the strain energy density at material point $k$. The force density vector is aligned with the relative position vector in the deformed configuration and it can be defined in the form
\begin{equation}
\mb{t}_{(k)(j)} = \frac{1}{2}A\frac{\mb{y}_{(j)}-\mb{y}_{(k)}}{|\mb{y}_{(j)}-\mb{y}_{(k)|}}
\label{eq:12}
\end{equation}
and
\begin{equation}
\mb{t}_{(j)(k)} = -\frac{1}{2}B\frac{\mb{y}_{(j)}-\mb{y}_{(k)}}{|\mb{y}_{(j)}-\mb{y}_{(k)|}},
\label{eq:13}
\end{equation}
where $A$ and $B$ are auxiliary parameters that are dependent on engineering material constants, deformation field, and the horizon.


\section{Derivation for general hyperelastic materials}\label{section:3}

Given the strain energy representation of a general hyperelastic material in continuum mechanics, we could obtain the explicit form of its peridynamics internal force density following some common procedures. Once we get the explicit form of the force density, we could further derive the peridynamics version of governing equations and proceed with the simulation.

The derivation procedures for a general hyperelastic material are listed below:
\begin{enumerate}
\item{Represent the strain energy density $W$ in terms of the internal force density $\mb{t}$, using the relationships in Equation \blue{\ref{eq:7}} \~{} \blue{\ref{eq:10}}. The strain energy densities in continuum mechanics are generally represented in terms of the stress tensor $\sigma$, or could be easily recast into this form. Then the derivation to the peridynamics form is straightforward with the relationships between $\sigma$ and $\mb{t}$.}
\item{Substitute the form of $\mb{t}$ (Equation \blue{\ref{eq:12}} and \blue{\ref{eq:13}}) into the representation of $W$ and perform differentiation using Equation \blue{\ref{eq:11}}. The resulting explicit expression of $\mb{t}$ contains several auxiliary parameters.}
\item{Determine the material parameters by equating the strain energy representations in continuum mechanics with their counterparts in peridynamics under several simple loading cases. The loading cases are carefully chosen so that the strain energy density could be analytically computed with the continuum mechanics representation. Then by equating the energy value with the peridynamics form of strain energy density, we could determine the peridynamics material parameters.}
\end{enumerate}

\section{Derivation for linear elastic materials}\label{section:4}

We choose to design our peridynamics constitutive model based on the linear elastic model in continuum mechanics due to its simplicity. More general hyperelastic models could be derived following the same procedures introduced above. For instance, Bang\blue{\cite{bang2016peridynamic}} performed derivations for the incompressible Neo-Hookean material.

\subsection{Strain energy density in terms of force density}

For an isotropic linear elastic material, the explicit form of the strain energy density, $W_{(k)}$, at material point $k$ can be represented as
\begin{equation}
W_{(k)} = \frac{\kappa}{2}\theta_{(k)}^2+\left[\frac{1}{4\mu}(\sigma_{xx(k)}^2+\sigma_{yy(k)}^2+\sigma_{zz(k)}^2)
                                         +\frac{1}{2\mu}(\sigma_{xy(k)}^2+\sigma_{xz(k)}^2+\sigma_{yz(k)}^2)
                                         -\frac{3\kappa^2}{4\mu}\theta_{(k)}^2
                                    \right],
\label{eq:14}
\end{equation}
in which $k$ and $\mu$ are the bulk modulus and the shear modulus. Note that we ignore the temperature change in the course of deformation. The first term and the second term at the right-hand side of the equation represent the dilatational energy and the distortional energy, respectively. The dilatation, $\theta_{(k)}$ is defined through
\begin{equation}
\theta_{(k)} = \epsilon_{xx(k)}+\epsilon_{yy(k)}+\epsilon_{zz(k)} = \frac{\sigma_{xx(k)}+\sigma_{yy(k)}+\sigma_{zz(k)}}{3\kappa}.
\label{eq:15}
\end{equation}
We rearrange the strain energy density to a slightly different form
\begin{equation}
\begin{aligned}
W_{(k)} =& \frac{\kappa}{2}\theta_{(k)}^2 -\frac{3\kappa^2}{4\mu}\theta_{(k)}^2 \\
         &+\frac{1}{8\mu}\left[(\sigma_{xx(k)}^2+\sigma_{xy(k)}^2+\sigma_{xz(k)}^2) + (\sigma_{xx(k)}^2+\sigma_{xy(k)}^2+\sigma_{xz(k)}^2)\right]\\
         &+\frac{1}{8\mu}\left[(\sigma_{yx(k)}^2+\sigma_{yy(k)}^2+\sigma_{yz(k)}^2) + (\sigma_{yx(k)}^2+\sigma_{yy(k)}^2+\sigma_{yz(k)}^2)\right]\\
         &+\frac{1}{8\mu}\left[(\sigma_{zx(k)}^2+\sigma_{zy(k)}^2+\sigma_{zz(k)}^2) + (\sigma_{zx(k)}^2+\sigma_{zy(k)}^2+\sigma_{zz(k)}^2)\right],
\end{aligned}
\label{eq:16}
\end{equation}
where each term involves stress component corresponds to the contribution from one of its immediate neighborhood $(k-l)$,$(k+l)$,$(k-m)$,$(k+m)$,$(k-n)$ and $(k+n)$.
Utilizing the relationship by Equation \blue{\ref{eq:10}} leads to the following result
\begin{equation}
\begin{aligned}
W_{(k)} =& (\frac{\kappa}{2} -\frac{3\kappa^2}{4\mu})\theta_{(k)}^2 \\
         &+\frac{1}{2\mu}\sum_{\substack {j=k-l,k+l,\\ \quad k-m,k+m,\\ \quad k-n,k+n}}(\mb{t}_{(k)(j)}|\mb{x}_{(j)}-\mb{x}_{(k)}|V_{(j)})\cdot(\mb{t}_{(k)(j)}|\mb{x}_{(j)}-\mb{x}_{(k)}|V_{(j)}).
\end{aligned}
\label{eq:17}
\end{equation}

We can also rewrite the dilatation, $\theta_{(k)}$, in a slightly different form as
\begin{equation}
\begin{aligned}
\theta_{(k)} =& \frac{\sigma_{xx(k)}+\sigma_{yy(k)}+\sigma_{zz(k)}}{3\kappa}\\
        =& \frac{1}{3\kappa}(\frac{1}{2}\sigma_{xx(k)}+\frac{1}{2}\sigma_{yy(k)}+\frac{1}{2}\sigma_{zz(k)}
         + \frac{1}{2}\sigma_{xx(k)}+\frac{1}{2}\sigma_{yy(k)}+\frac{1}{2}\sigma_{zz(k)}).
\end{aligned}
\label{eq:18}
\end{equation}
Substitute Equation \blue{\ref{eq:9}} into the above equation, we get the explicit form of $\theta_{(k)}$ in terms of $\mb{t}_{(k)}$:
\begin{equation}
\theta_{(k)} = \frac{1}{3\kappa}\left(\sum_{\substack {j=k-l,k+l,\\ \quad k-m,k+m,\\ \quad k-n,k+n}}(\mb{t}_{(k)(j)}\cdot(\mb{x}_{(j)}-\mb{x}_{(k)})V_{(j)})\right).
\label{eq:19}
\end{equation}

Combining Equation \blue{\ref{eq:17}} and Equation \blue{\ref{eq:19}} leads to the representation of strain energy density $W_{(k)}$ in terms of internal force density $\mb{t}_{(k)(j)}$.

\subsection{Explicit form of force density}



\subsection{Determination of material parameters}

\bibliographystyle{eg-alpha}
\bibliography{../references}

\end{document}
