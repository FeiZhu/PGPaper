% ---------------------------------------------------------------------
% EG author guidelines plus sample file for EG publication using LaTeX2e input
% D.Fellner, v1.17, Sep 23, 2010


\title[Peridynamics-Based Fracture Animation for Elastoplastic Solids]%
      {Peridynamics-Based Fracture Animation for Elastoplastic Solids}

% for anonymous conference submission please enter your SUBMISSION ID
% instead of the author's name (and leave the affiliation blank) !!
\author[W. Chen et al.]{W. Chen, F. Zhu, J. Zhao,
  S. Li, G. Wang}

%\author[D. Fellner \& S. Behnke]
%       {D.\,W. Fellner\thanks{Chairman Eurographics Publications Board}$^{1,2}$
%        and S. Behnke$^{2}$
%%        S. Spencer$^2$\thanks{Chairman Siggraph Publications Board}
%        \\
%% For Computer Graphics Forum: Please use the abbreviation of your first name.
%         $^1$TU Darmstadt \& Fraunhofer IGD, Germany\\
%         $^2$Institut f{\"u}r ComputerGraphik \& Wissensvisualisierung, TU Graz, Austria
%%        $^2$ Another Department to illustrate the use in papers from authors
%%             with different affiliations
%       }

% ------------------------------------------------------------------------

% if the Editors-in-Chief have given you the data, you may uncomment
% the following five lines and insert it here
%
% \volume{27}   % the volume in which the issue will be published;
% \issue{1}     % the issue number of the publication
% \pStartPage{1}      % set starting page

%-------------------------------------------------------------------------
\begin{document}

\maketitle

%-------------------------------------------------------------------------
\section{Revision Summary}

We have carefully revised our submission according to the suggestions of the associate editor. The major changes to the manuscript include (but not limited to):
\begin{enumerate}
\item{Description of the damping models is added to the manuscript along with corresponding damping parameters in Table 1, which would be helpful to reproduce the results.}

\item{Detailed discussion for picking $\delta$ values is included and the new beam demo is added to our final video.
      Here, we emphasize that the simulation plausibility is not very
      sensitive to the parameter $\delta$, and a value range from $1.0\lambda$ to $2.0\lambda$ is recommended.
      As for the seeming odd $\delta$ values ($1.34\lambda, 1.38\lambda$), these values are fine-tuned just to achieve our best-synchronized results with FEM. We additionally provide three more supplemental videos that use other values($1.0\lambda,1.2\lambda,1.3\lambda$), in which the simulations are still physically plausible but are less-synchronized with FEM compared to the fine-tuned results in our final video.}

\item{Discussion on pros and cons of our method compared with \cite{Pauly:2005:MAF:1073204.1073296} is included.}
\end{enumerate}

Please refer to our revised paper.
%-------------------------------------------------------------------------
\bibliographystyle{eg-alpha}
%\bibliographystyle{eg-alpha-doi}

\bibliography{../references}

%-------------------------------------------------------------------------
%\newpage

\end{document}
