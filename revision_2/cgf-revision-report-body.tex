% ---------------------------------------------------------------------
% EG author guidelines plus sample file for EG publication using LaTeX2e input
% D.Fellner, v1.17, Sep 23, 2010


\title[Peridynamics-Based Fracture Animation for Elastoplastic Solids]%
      {Peridynamics-Based Fracture Animation for Elastoplastic Solids}

% for anonymous conference submission please enter your SUBMISSION ID
% instead of the author's name (and leave the affiliation blank) !!
\author[W. Chen et al.]{W. Chen, F. Zhu, J. Zhao,
  S. Li, G. Wang}

%\author[D. Fellner \& S. Behnke]
%       {D.\,W. Fellner\thanks{Chairman Eurographics Publications Board}$^{1,2}$
%        and S. Behnke$^{2}$
%%        S. Spencer$^2$\thanks{Chairman Siggraph Publications Board}
%        \\
%% For Computer Graphics Forum: Please use the abbreviation of your first name.
%         $^1$TU Darmstadt \& Fraunhofer IGD, Germany\\
%         $^2$Institut f{\"u}r ComputerGraphik \& Wissensvisualisierung, TU Graz, Austria
%%        $^2$ Another Department to illustrate the use in papers from authors
%%             with different affiliations
%       }

% ------------------------------------------------------------------------

% if the Editors-in-Chief have given you the data, you may uncomment
% the following five lines and insert it here
%
% \volume{27}   % the volume in which the issue will be published;
% \issue{1}     % the issue number of the publication
% \pStartPage{1}      % set starting page

%-------------------------------------------------------------------------
\begin{document}

\maketitle

%-------------------------------------------------------------------------
\section{Revision Summary}

We have carefully revised our submission according to the suggestions of the reviewers. The major changes include (but not limited to):
\begin{enumerate}
\item{Blablabla.}
\end{enumerate}

\section{Answers to Reviewers' Comments}

Here we respond to the comments made by each reviewer.

\noindent{}\textbf{Reviewer \#1:}

\noindent{Q: }\emph{More explicit discussion on damping in the paper is preferred.}

Blablabla.

\noindent{}\textbf{Reviewer \#2:}

\noindent{Q: }\emph{Limitations for meshing and explicit fracture comparison to other methods.}

Blablabla.

\noindent{}\textbf{Reviewer \#3:}

\noindent{Q: }\emph{How much does the interaction radius parameter $\delta$ affect the simulation quality?}

Blablabla.

\noindent{Q: }\emph{There are missing steps for discretization. How do we go from equation 1 to discrete equations of motion, where $\mathbf{x}$ and $\mathbf{x'}$ are particles rather than arbitrary points in the material? }

Blablabla.

\noindent{Q: }\emph{Can't the use of the weights defined in equation 4 be interpreted as just a discretization of divergence, except over the entire horizon rather than over an element neighborhood?}

To our minds, we do not think the weight function in Peridynamics amounts to a discretization of divergence in FEM.
In fact, by comparing the governing equations in both theories, we could easily derive that
the divergence of stress in FEM is equivalent to integration of force density over horizon, which is
\begin{equation}
\nabla\cdot\sigma = \int_{H_\mathbf{x}}[\mathbf{T}\langle\mathbf{x}',\mathbf{x}\rangle - \mathbf{T}\langle\mathbf{x},\mathbf{x}'\rangle]dV_{\mathbf{x}'}
\end{equation}
From discretization, we could naturally have
\begin{equation}
\nabla\cdot\sigma = \sum_{H_\mathbf{x}}[\mathbf{T}\langle\mathbf{x}',\mathbf{x}\rangle - \mathbf{T}\langle\mathbf{x},\mathbf{x}'\rangle]V_{\mathbf{x}'}
\end{equation}
We note that it is the force summation within horizon that should be interpreted as a discretization of divergence.
Although weight function are encoded in force density, $\omega$ is used only to compute the force summation in a weighted way, which is inherently different from other methods such as SPH where weight functions are simutaneously used to calculate quantities' first-order or second-order derivatives
(e.g. $\nabla\cdot\mathbf{A_i} = -\frac{1}{\rho_i}\sum_j(\mathbf{A_i - A_j})\cdot\nabla\omega_{ij}$)

That is, in our formulation, what involved here are just displacements of material points,
instead of the spatial derivative of the displacements, which consequently lead to the biggest difference between Peridynamic and other mesh(FEM) or meshless method.

In addition, $\omega$ is only associated with positions in reference space, rather than positions in deformed space,
so we could pre-gurantee its value rationality by appropriately populating our material points in advance.

\noindent{Q: }\emph{Correction for typos.}

Blablabla.

\noindent{Q: }\emph{Lack discussion for XFEM-based methods in the related work.}

Blablaba.

\noindent{Q: }\emph{How is equation 17 used, exactly? what are "rest\_bond\_num" and "initial\_bond\_num" in the definition of $\phi$?}

Blablabla.

\noindent{Q: }\emph{Unreasonable statements on geometric linearity and stress-strain linearity.}

Blablabla.

\noindent{Q: }\emph{How many elements were used for each example? How did the performance compare to the performance of peridynamics?}

Blablabla.

\noindent{Q: }\emph{Question on the supplemental video (random cube)?}

Blablabla.

%-------------------------------------------------------------------------
\bibliographystyle{eg-alpha}
%\bibliographystyle{eg-alpha-doi}

%\bibliography{../references}

%-------------------------------------------------------------------------
%\newpage

\end{document}
