% ---------------------------------------------------------------------
% EG author guidelines plus sample file for EG publication using LaTeX2e input
% D.Fellner, v1.17, Sep 23, 2010


\title[Peridynamics-Based Fracture Animation for Elastoplastic Solids]%
      {Peridynamics-Based Fracture Animation for Elastoplastic Solids}

% for anonymous conference submission please enter your SUBMISSION ID
% instead of the author's name (and leave the affiliation blank) !!
\author[paperID]{paperID}

%\author[D. Fellner \& S. Behnke]
%       {D.\,W. Fellner\thanks{Chairman Eurographics Publications Board}$^{1,2}$
%        and S. Behnke$^{2}$
%%        S. Spencer$^2$\thanks{Chairman Siggraph Publications Board}
%        \\
%% For Computer Graphics Forum: Please use the abbreviation of your first name.
%         $^1$TU Darmstadt \& Fraunhofer IGD, Germany\\
%         $^2$Institut f{\"u}r ComputerGraphik \& Wissensvisualisierung, TU Graz, Austria
%%        $^2$ Another Department to illustrate the use in papers from authors
%%             with different affiliations
%       }

% ------------------------------------------------------------------------

% if the Editors-in-Chief have given you the data, you may uncomment
% the following five lines and insert it here
%
% \volume{27}   % the volume in which the issue will be published;
% \issue{1}     % the issue number of the publication
% \pStartPage{1}      % set starting page


%-------------------------------------------------------------------------
\begin{document}

% \teaser{
%  \includegraphics[width=\linewidth]{eg_new}
%  \centering
%   \caption{New EG Logo}
% \label{fig:teaser}
% }

\maketitle

\begin{abstract}
In this paper, we exploit the use of peridynamics theory for graphical animation of material deformation and fracture. We present a new meshless framework for elastoplastic constitutive modeling that contrasts with previous approaches in graphics. Our peridynamics-based elastoplasticity model represents deformation behaviors of materials with high realism. We validate our model using various material properties and comparisons with FEM simulations. Besides, the integral-based nature of peridynamics makes it trivial to model material discontinuities, which outweighs differential-based methods in both accuracy and ease of implementation. We propose a simple strategy to model fracture in the setting of peridynamics discretization. We demonstrate that the fracture criterion combined with our elastoplasticity model could realistically produce ductile fracture as well as brittle fracture. Our work is the first application of peridynamics in graphics that could create a wide range of material phenomena including elasticity, plasticity, and fracture. The complete framework provides an attractive alternative to existing methods for producing modern visual effects.

\begin{classification} % according to http://www.acm.org/class/1998/
\CCScat{Computer Graphics}{I.3.7}{Three-Dimensional Graphics and Realism}{Animation}
\end{classification}

\end{abstract}





%-------------------------------------------------------------------------
\section{Introduction}

%-------------------------------------------------------------------------
\section{Related Work}

%-------------------------------------------------------------------------
\section{Background}

%-------------------------------------------------------------------------
\section{Overview}

%-------------------------------------------------------------------------
\section{Elastoplastic Model}

%-------------------------------------------------------------------------
\section{Fracture}

%-------------------------------------------------------------------------
\section{Implementation}

%-------------------------------------------------------------------------
\section{Results}

%-------------------------------------------------------------------------
\section{Discussion}

%-------------------------------------------------------------------------
%\bibliographystyle{eg-alpha}
%\bibliographystyle{eg-alpha-doi}

%\bibliography{references}

%-------------------------------------------------------------------------
%\newpage

\end{document}
