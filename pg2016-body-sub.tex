% ---------------------------------------------------------------------
% EG author guidelines plus sample file for EG publication using LaTeX2e input
% D.Fellner, v1.17, Sep 23, 2010


\title[Peridynamics-Based Fracture Animation for Elastoplastic Solids]%
      {Peridynamics-Based Fracture Animation for Elastoplastic Solids}

% for anonymous conference submission please enter your SUBMISSION ID
% instead of the author's name (and leave the affiliation blank) !!
\author[paper1047]{paper1047}

%\author[D. Fellner \& S. Behnke]
%       {D.\,W. Fellner\thanks{Chairman Eurographics Publications Board}$^{1,2}$
%        and S. Behnke$^{2}$
%%        S. Spencer$^2$\thanks{Chairman Siggraph Publications Board}
%        \\
%% For Computer Graphics Forum: Please use the abbreviation of your first name.
%         $^1$TU Darmstadt \& Fraunhofer IGD, Germany\\
%         $^2$Institut f{\"u}r ComputerGraphik \& Wissensvisualisierung, TU Graz, Austria
%%        $^2$ Another Department to illustrate the use in papers from authors
%%             with different affiliations
%       }

% ------------------------------------------------------------------------

% if the Editors-in-Chief have given you the data, you may uncomment
% the following five lines and insert it here
%
% \volume{27}   % the volume in which the issue will be published;
% \issue{1}     % the issue number of the publication
% \pStartPage{1}      % set starting page


%-------------------------------------------------------------------------
\begin{document}

\teaser{
 \includegraphics[width=\linewidth]{eg_new}
 \centering
  \caption{here is ball crack wall, brittle and ductile, isotropic and anisotropic}
\label{fig:teaser}
}

\maketitle

\begin{abstract}
In this paper, we exploit the use of peridynamics theory for graphical animation of material deformation and fracture. We present a new meshless framework for elastoplastic constitutive modeling that contrasts with previous approaches in graphics. Our peridynamics-based elastoplasticity model represents deformation behaviors of materials with high realism. We validate the model by varying the material properties and performing comparisons with FEM simulations. Besides, the integral-based nature of peridynamics makes it trivial to model material discontinuities, which outweighs differential-based methods in both accuracy and ease of implementation. We propose a simple strategy to model fracture in the setting of peridynamics discretization. We demonstrate that the fracture criterion combined with our elastoplasticity model could realistically produce ductile fracture as well as brittle fracture. Our work is the first application of peridynamics in graphics that could create a wide range of material phenomena including elasticity, plasticity, and fracture. The complete framework provides an attractive alternative to existing methods for producing modern visual effects.

\begin{classification} % according to http://www.acm.org/class/1998/
\CCScat{Computer Graphics}{I.3.7}{Three-Dimensional Graphics and Realism}{Animation}
\end{classification}

\end{abstract}





%-------------------------------------------------------------------------
\section{Introduction}

The simulation of deformable materials has been an important research topic in computer graphics for decades, since the early work by Terzopoulos and colleagues \cite{Terzopoulos:1987:EDM:37402.37427}. One of the strongest driving forces behind the active research is the persistently growing needs for higher realism from the visual effects industry. Materials in real-world exhibit complex behaviors, such as coupled elastoplastic deformations, fracture, etc. The complicated material behaviors are difficult to be virtually replicated by any single method despite the numerous ones that have been developed thus far. Existing approaches generally excel at some phenomena but would stumble (if not fail) at others. For instance, mesh-based methods \cite{Muller:2004:IVM:1006058.1006087,Irving:2004:IFE:1028523.1028541,Teran:2005:RQF:1073368.1073394,Sifakis:2012:FSD:2343483.2343501} are a good choice to simulate elastic deformations whereas not preferred for phenomena that involve topological changes. Particle-based methods \cite{Muller:2003:PFS:846276.846298,Pauly:2005:MAF:1073204.1073296,Stomakhin:2013:MPM:2461912.2461948} are considered suitable for modeling topological changes, however the inherent loss of connectivity information would cause undesirable numerical fracture \cite{Liu:2011:AKM:2065362.2066108, Zhu:2016MPM} while simulating large deformations.

We build on recent developments of peridynamics theory in the computational physics community \cite{Silling2000,silling2007peridynamic,mitchell2011nonlocal, emmrich2013peridynamics,madenci2014peridynamic} and propose a novel framework for graphical animation of varied deformation behaviors and fracture. Our aim is to enrich available options of simulation techniques for easier and better animation production. Peridynamics was first adopted to animation applications by Levine et al. \cite{Levine:2015:PPS:2849517.2849526}. They described a simple spring-mass system to handle brittle fracture of solids. In contrast, we handle elastoplasticiy, brittle fracture, and ductile fracture in a single framework. To this end, we propose several novel contributions in this work. We first present an elastoplastic constitutive model in the peridynamics-based framework with simple extension to anisotropy, and the model is validated against results produced by FEM. Furthermore, we show that both brittle and ductile fracture phenomena can be naturally represented with nearly no effort by integrating a simple fracture criterion into this material model. This is due to the integral-based formulation of peridynamics, in which forces at a material point are computed by gathering contributions from all material points in its interaction range through integration. On the other hand, methods based on classical continuum mechanics formulate force computations with partial differential equations that fail to be applicable directly on singularities such as a crack. This feature makes our peridynamics-based framework more attractive over existing approaches for producing animations that involve fracture. Lastly, our method is simple to implement and trivially parallelizable, providing a useful alternative to previous methods for animation production.

%-------------------------------------------------------------------------
\section{Related Work}

A large body of literature has been devoted to physical simulation of natural phenomena as a result of active research. A complete literature review is beyond the scope of this paper. In the following we comment only on the representative works most related to ours.

\noindent{\textbf{Elastoplasticity Animation}}~The modeling of deformable plasticity in graphics dates back to the pioneering work by Terzopoulos and Fleischer \cite{Terzopoulos:1988:MID:378456.378522}. O'Brien and colleagues \cite{O'Brien:2002:GMA:566654.566579} incorporated a similar additive plasticity model into a finite element simulation to animate ductile fracture. The strain measure was decomposed into two components, where one is due to elastic deformation and the other due to plastic deformation. M\"{u}ller et al. \cite{Muller:2004:PBA:1028523.1028542} applied this model in their point-based animation framework and simulated plastic behaviors of objects.  Irving et al. \cite{Irving:2004:IFE:1028523.1028541} presented a multiplicative formulation of plasticity and pointed out that their model was better handling finite plastic deformation than the additive one. In contrast to the additive model, they decomposed the deformation gradient into two components through multiplication. The multiplicative model was extensively used by later works to animate phenomena that involve plasticity. Bargteil et al. simulated large viscoplastic flow\cite{Bargteil:2007:FEM:1276377.1276397}, Gerszewski and his colleagues animated elastoplastic solids\cite{Gerszewski:2009:PMA:1599470.1599488}, and Stomakhin et al. modeled plasticity of snow\cite{Stomakhin:2013:MPM:2461912.2461948}, just to name a few. Unfortunately, neither of the above plasticity models applies in the peridynamics framework because there is no concept of strain nor deformation gradient in the integral-based formulation. As a result, we present a new constitutive model for peridynamics in this work to animate elastoplastic solids.

\noindent{\textbf{Fracture Animation}}~Numerous methods have been proposed on fracture animation because the stunning phenomenon of fracture and failure is an indispensable visual element in animated movies and video games. 

\noindent{\textbf{Peridynamics}}

%-------------------------------------------------------------------------
\section{Background}

\emph{This section describes background of peridynamics theory, and conceptual comparsion with traditional particle methods.}

%-------------------------------------------------------------------------
\section{Overview}

\emph{This section briefly describes the algorithm pipeline.}

%-------------------------------------------------------------------------
\section{Elastoplastic Model}

\emph{This section introduces the constitutive model.}

%-------------------------------------------------------------------------
\section{Fracture}

\emph{This section describes the fracture criterion, and the mesh split strategy.}

%-------------------------------------------------------------------------
\section{Implementation}

\emph{This section provides some implementation details. The mesh embedding technique, the collision handling, and rendering.}

%-------------------------------------------------------------------------
\section{Results}

\emph{This section demonstrates some results.}

%-------------------------------------------------------------------------
\section{Discussions}

\emph{This section concludes the paper, and discusses the limitations.}

%-------------------------------------------------------------------------
\bibliographystyle{eg-alpha}
%\bibliographystyle{eg-alpha-doi}

\bibliography{references}

%-------------------------------------------------------------------------
%\newpage

\end{document}
